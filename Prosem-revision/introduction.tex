\chapter{Introduction}
\label{sec::introduction}

Abstract:

Remote Working has gained a lot of importance last few years, primarily due to the pandemic forcing many employees out of their offices. This lead to many companies adapting to the new circumstances and transferring to remote work, and as the employees adapted as well, the upsides of remote working became very clear: no commute times, flexible working hours, and a big decrease in costs for office maintenance like cleaning and electricity, just to name a few. In short, it is a clear case that Remote Work is here to stay. However, everything has its downsides: Due to the heavy reliance on 2D video conferencing tools like Zoom, social interactions and meetings become drastically less efficient, because they cannot reproduce nonverbal behavior like eye contact or gestures, which makes these interactions less natural. This is where Augmented Reality (AR) comes in. With AR technology improving at very fast speeds, it has the potential to solve many problems that emerged with remote work, especially the social aspect. In this paper, I am going to explore how the design of a virtual avatar and its interaction with the real world impacts the social effects it has on the user.


\section{What is Augmented Reality?}

A topic that gathered more attention than any other in the current tech space is Augmented Reality or AR for short. In its most basic definition, this technology allows the user to project virtual objects onto the real-world 3d space and allows for interaction between real people and a virtual environment, that can contain everything from simple overlays to complex structures and virtual worlds. \cite{Carmigniani:2011te}

The time when Augmented Reality was just a fancy demo is long gone, it has been playing a big part in many people's everyday lives, and its influence is projected to only get bigger in the future. AR is built into many current technologies, from entertainment purposes like the captivating phenomenon that was the mobile game "Pokemon Go", where AR played a big part in the game’s rise to popularity, to useful everyday appliances like the heads-up displays that are built into many modern cars. AR is so useful in fact, that it is projected to be one of the fastest-growing industries in the next decade \cite{XR_Projection}. These were only some of the applications in the private sector, but currently, the biggest upsides for AR are in the industrial sector. Many big companies invested billions of dollars to develop the most cutting-edge AR technologies and to gain an early lead in their market. Just to name the biggest competitors, there is Microsoft with the HoloLens, Meta with their rendition of AR glasses, and Apple, which is rumored to develop an AR product of their own. It is safe to say that there is a lot of potential in augmented Reality when every single one of the biggest players in tech decides to make AR a priority in its development and invests billions in it. 

However, AR is not limited to a particular type of display technology nor only to visual overlays to the real world. AR can just as well be applied to all senses like smelling or hearing, but it is less popular in these forms and has not found many appliances yet. That is why in this paper the focus lies on the visual aspect of AR, especially on wearable AR headsets like the Microsoft Hololens, where the user wears a headset that computes the spacial position and rotation of the user's head by tracking the room with cameras and superimpose virtual content onto the real world by displaying it right in front of the user's eyes.

\section{The importance of Social Presence}
A pleasant user experience in a virtual environment depends on several factors, one of which is the perceived presence of the user, meaning that the user feels connected to the virtual environment and the interactions feel authentic and natural \cite{PresencePaper}. By introducing social interactions into the virtual environment, social presence becomes very important as well. The social presence of virtual characters plays a key role in how effective AR will be in replacing 2D video conferencing. It can be loosely defined as the sense of "being together" with another person through a virtual medium. It is important for social presence to be perceived as high because as many studies confirmed, a high social presence leads to greater satisfaction in the collaborators and improved quality in social interactions, which in turn leads to improved performance, which will be discussed further in the main part. In remote collaborations through AR, a collaborator's virtual embodiment is referred to as their avatar and the virtual embodiment of virtual characters are called virtual agents. The realism of these avatars and agents greatly influences the perceived social presence as well, and in the main part, we will discuss what the important variables in designing virtual characters are to maximize their perceived social presence. \cite{10.1117/12.387188}
Unlike in Virtual Reality (VR), where the user is fully immersed in a virtual world, AR adds a virtual overlay onto the real world, and studies show that this creates a high degree of social presence which makes it highly beneficial for remote collaboration.

\section{Why remote work is here to stay}

Even before covid, remote work was on the rise because many costs can be saved by working from home. Studies by Raffael Ferreira et al. found that remote work done right improves the work-life balance, and schedule flexibility of workers while greatly reducing costs for companies. Costs like travel expenses and catering can be saved and a cheaper workforce from other countries can be hired without complications. However, the study also finds that by working remotely many problems can occur: communication between workers suffers due to the collaborators not being able to read the body language and less visual contact in general \cite{joitmc7010070}. These problems can be addressed with the help of AR and technologies that will be shown in this paper.


